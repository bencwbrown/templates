%% This is a sample for a multi-section exam for exams.cls 1.9b.
%
\documentclass{exams}
% 
% If you want \Bbb or \frak
\usepackage{amsfonts}
%
% Use other packages as needed.
% Packages compatible with article class ought to work OK.


%%
%% Process your file twice so that the format can 
%% put the CONTINUED and END on the correct pages.
%%


%%
%% This data generates the first page header and 
%% the runninghead. Replace with the correct data!
%%
\examcode{MA 2999}
\examyear{2}   % 1-4 for UG, 5=MSc
\examdate{APRIL 1999}
\examtitle{DIFFERENTIAL EQUATIONS A+B} % In CAPS
\examtime{3}   % This is the total length of exam in hours.
               %
\multisection  % must use this for a multisection exam. 
               % See below how to title the sections.
               %
\marks         % use this to add a comment to the rubric about 
               % marks in margin.
               %
%\resit         % Use this for a resit exam.
%\calculators   % Uncomment this if calculators are allowed.


%% 
%% You can label parts of a question using
%% the enumerate environment. Adjust its appearance 
%% as follows if you prefer roman numerals for the
%% parts. The default is lower case letters a), b),...
%%
\renewcommand{\theenumi}{\alph{enumi}}
\renewcommand{\labelenumi}{\theenumi)}
%% If you need more than one level of enumerates
%% then define the format for those, too.
\renewcommand{\theenumii}{\roman{enumii}}
%%

%%
%% Define your own macros, e.g.
%%
\newcommand{\R}{\mathbb{R}}
\newcommand{\C}{\mathbb{C}}
\newcommand{\N}{\mathbb{N}}
\newcommand{\Q}{\mathbb{Q}}
\newcommand{\Z}{\mathbb{Z}}
\newcommand{\g}{\mathfrak{g}}


\begin{document}

\begin{exam}

%\rubricsection{The title of the section}
%        {time to spend on this section}
%        {THE RUBRIC}

%A section with a user-defined rubric
\rubricsection{SECTION A: Theory of Whatever}{1 hour}{%
ANSWER ALL QUESTIONS IN THIS SECTION\smallskip

For each question answer BOTH parts.}

\begin{question}
\begin{enumerate}
\item Let $f \colon \R^n \to \R^m$ be a map. Say what it means for $f$ to be
\textit{differentiable} at $a \in \R^n$. If $f \colon \R^n \to \R$ and
$g \colon \R^n \to \R$ are both differentiable at $a$, prove Leibniz'
Rule that the pointwise product $h(x) = f(x)g(x)$ is differentiable at
$a$ and give the formula for the derivative $D_ah$ in terms of $D_af$
and $D_ag$.\mark{20}

\item For each of the following functions determine whether they are
differentiable at the origin $(0,0)$. In each case give reasons for your
answer (you may quote any relevant result from the course).

\begin{enumerate}
\item $f(x,y) = x^2 + 3x^2y + (y-1)^3$;\mark{3}
\item $f(x,y) = \sqrt{x^2+y^2}$.\mark{2}
\end{enumerate}
\end{enumerate}
\end{question}

\newpage

\begin{question}
\begin{enumerate}
\item Let $f \colon \R^n \to \R^m$ be a map. Say what it means for $f$ to be
\textit{differentiable} at $a \in \R^n$. If $f \colon \R^n \to \R$ and
$g \colon \R^n \to \R$ are both differentiable at $a$, prove Leibniz'
Rule that the pointwise product $h(x) = f(x)g(x)$ is differentiable at
$a$ and give the formula for the derivative $D_ah$ in terms of $D_af$
and $D_ag$.\mark{20}

\item For each of the following functions determine whether they are
differentiable at the origin $(0,0)$. In each case give reasons for your
answer (you may quote any relevant result from the course).

\begin{enumerate}
\item $f(x,y) = x^2 + 3x^2y + (y-1)^3$;\mark{3}
\item $f(x,y) = \sqrt{x^2+y^2}$.\mark{2}
\end{enumerate}
\end{enumerate}
\end{question}

%\section{The title of the section}
%        {time to spend on this section}
%        {number of questions to answer in this section}

\section{SECTION B: Another theory}{1 hour}{2}

%% Marks can be displayed for parts of a question using the \mark{}
%% macro at the end of a line where the marks should appear. Add
%% \marks to the preamble to put a comment on this in the rubric.
%% A warning will be issued if you use \mark{} without adding 
%% \marks to the preamble.

\begin{question}
\begin{enumerate}
\item Describe the Euclidean algorithm for finding the
highest common factor of two integers.\mark{5}
\item Prove that every natural number $n$ can be expressed
uniquely in the form
$$
n=p_1^{a_1}\cdots p_r^{a_r}
$$
where $p_1,\ldots,p_r$ are primes in strictly increasing order
and $a_1,\ldots,a_r$ are positive integers.\mark{12}
\item Factorize 22499 into primes using the fact that it has a
prime factor in common with 60551.
\mark{8}
\end{enumerate}
\end{question}

% The following command outputs CONTINUED and then starts a new page.
% Use this if you want to force a question to start on the next page.
\newpage

\begin{question}
Define the {\it M\"obius function} $\mu \colon \N \to \Z$
and state, without proof, the M\"obius inversion formula.\mark{5}

For $n \in \N$, let $\varphi(n)$ denote the {\it Euler
function}:-- the number of positive integers less than $n$ and coprime
to $n$. Prove that $n= \displaystyle{\sum_{d\vert n}} \varphi(d)$ and
deduce that $\varphi(n) = n\displaystyle{\sum_{d\vert n}} \mu(d)/d$.
By computing $\displaystyle{\sum_{d\vert n}} \varphi(d)$ when $n$ is a
prime power and using the multiplicative property of $\varphi$ derive
an explicit formula for $\varphi(n)$ in terms of the distinct prime
divisors of $n$.\mark{12}

Evaluate $\varphi(204800)$.\mark{8}
\end{question}

\begin{question}
\begin{enumerate}
\item Let $f \colon \R^n \to \R^m$ be a map. Say what it means for $f$ to be
\textit{differentiable} at $a \in \R^n$. If $f \colon \R^n \to \R$ and
$g \colon \R^n \to \R$ are both differentiable at $a$, prove Leibniz'
Rule that the pointwise product $h(x) = f(x)g(x)$ is differentiable at
$a$ and give the formula for the derivative $D_ah$ in terms of $D_af$
and $D_ag$.\mark{15}

\item For each of the following functions determine whether they are
differentiable at the origin $(0,0)$. In each case give reasons for your
answer (you may quote any relevant result from the course).

\begin{enumerate}
\item $f(x,y) = x^2 + 3x^2y + (y-1)^3$;\mark{5}
\item $f(x,y) = \sqrt{x^2+y^2}$.\mark{5}
\end{enumerate}
\end{enumerate}
\end{question}


%% 
%#1 Title of section
%#2 Length of exam
%#3 Number of questions for full credit
%#3 = number of optional  questions
%#4 = number of marks for compulsory qu.1
%#5 = number of marks for each optional  question

\xsection{SECTION C: Differential Equations B}{1 hour}{2}{30}{10}

\begin{question}
\begin{enumerate}
\item Describe the Euclidean algorithm for finding the
highest common factor of two integers.\mark{5}
\item Prove that every natural number $n$ can be expressed
uniquely in the form
$$
n=p_1^{a_1}\cdots p_r^{a_r}
$$
where $p_1,\ldots,p_r$ are primes in strictly increasing order
and $a_1,\ldots,a_r$ are positive integers.\mark{12}
\item Factorize 22499 into primes using the fact that it has a
prime factor in common with 60551.
\mark{13}
\end{enumerate}
\end{question}

\begin{question}
Define the {\it M\"obius function} $\mu \colon \N \to \Z$
and state, without proof, the M\"obius inversion formula.\mark{2}

For $n \in \N$, let $\varphi(n)$ denote the {\it Euler
function}:-- the number of positive integers less than $n$ and coprime
to $n$. Prove that $n= \displaystyle{\sum_{d\vert n}} \varphi(d)$ and
deduce that $\varphi(n) = n\displaystyle{\sum_{d\vert n}} \mu(d)/d$.
By computing $\displaystyle{\sum_{d\vert n}} \varphi(d)$ when $n$ is a
prime power and using the multiplicative property of $\varphi$ derive
an explicit formula for $\varphi(n)$ in terms of the distinct prime
divisors of $n$.\mark{4}

Evaluate $\varphi(204800)$.\mark{4}
\end{question}

\begin{question}
\begin{enumerate}
\item Let $f \colon \R^n \to \R^m$ be a map. Say what it means for $f$ to be
\textit{differentiable} at $a \in \R^n$. If $f \colon \R^n \to \R$ and
$g \colon \R^n \to \R$ are both differentiable at $a$, prove Leibniz'
Rule that the pointwise product $h(x) = f(x)g(x)$ is differentiable at
$a$ and give the formula for the derivative $D_ah$ in terms of $D_af$
and $D_ag$.\mark{5}

\item For each of the following functions determine whether they are
differentiable at the origin $(0,0)$. In each case give reasons for your
answer (you may quote any relevant result from the course).

\begin{enumerate}
\item $f(x,y) = x^2 + 3x^2y + (y-1)^3$;\mark{3}
\item $f(x,y) = \sqrt{x^2+y^2}$.\mark{2}
\end{enumerate}
\end{enumerate}
\end{question}

% \end{exam} will automatically print END after the last question.
\end{exam}


\end{document}
